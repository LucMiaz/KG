%change main font with XeLaTeX
\usepackage{fontspec}
\setmainfont[Ligatures=TeX]{Georgia}\newfontfamily\Georgia{Georgia}\setsansfont{Helvetica Neue}\newfontfamily\Helvetica{Helvetica Neue}
\setmonofont{Courier New}
%format des sections et sous-sections
\usepackage{sectsty}
\allsectionsfont{\sffamily}
\usepackage{titlesec}
\renewcommand{\thesection}{\sffamily\arabic{section}}
\renewcommand{\thesubsection}{\thesection.\arabic{subsection}}
\renewcommand{\thesubsubsection}{\thesubsection.\arabic{subsubsection}}
\renewcommand{\theparagraph}{\alph{paragraph}.}\titleformat*{\section}{\Large\bfseries}

%font
%\usepackage{newcent}%\usepackage{yfonts}%\usepackage{amsfonts}%\usepackage{MnSymbol}
 
%en-têtes et pieds de page
\usepackage{fancyhdr}
\fancypagestyle{body}{
\fancyhead[L]{Analysis of Rail Lubrification}
\fancyhead[R]{\thepage}
\fancyhead[C]{}
\fancyfoot[R]{}
\fancyfoot[C]{}
\renewcommand{\headrulewidth}{.25pt}
\renewcommand\headheight{14pt}}

\usepackage{calc}
\usepackage[a4paper,left=2.5cm,width=16.5cm,right=2.5cm, top=2.5cm, bottom=2.5cm]{geometry}
\usepackage{amsmath}

%entêtetitre
\fancypagestyle{title}{\fancyhead[L]{}\fancyhead[R]{}\fancyfoot[R]{}\fancyfoot[C]{}\fancyhead[C]{}\renewcommand{\headrulewidth}{0pt}}
%import csv as table
\usepackage{pgfplotstable}
\usepackage{booktabs, colortbl}
%french quotes
\usepackage{csquotes}

%ajoute des couleurs
\usepackage{xcolor}
\definecolor{bleurusse}{HTML}{5A8BBA}
\definecolor{bleu}{HTML}{D8B365}
\definecolor{orangerusse}{HTML}{E84E0F}
\definecolor{rouge2}{HTML}{E84E0F}
\definecolor{orangebrewer}{RGB}{226,115,0}
\definecolor{gris}{HTML}{F5F5F5}
\definecolor{orange}{HTML}{E66101}
\definecolor{vert}{HTML}{7DE023}
\definecolor{violette}{HTML}{7B3294}
\definecolor{brun}{HTML}{5AB4AC}
\definecolor{rouge}{HTML}{D14316}
\definecolor{jaune}{HTML}{FFFF33}
\definecolor{darkgrey}{HTML}{272822}


%change la marge des légendes:
\usepackage{caption, xcolor,tabu}
\colorlet{captionlabel}{bleurusse}

\captionsetup{
  font=footnotesize,
  format=hang,
  singlelinecheck=false,
  labelfont={bf, color=captionlabel}
}

%signes et style de listes et enumerations
\renewcommand{\labelitemi}{$\bullet$}
\renewcommand{\labelitemiii}{$\diamond$}
\renewcommand{\labelitemii}{$\cdot$}
\renewcommand{\labelitemiv}{$\ast$}
%
\renewcommand{\labelenumi}{(\arabic{enumi}.}
\renewcommand{\labelenumii}{\alph{enumii})}
\usepackage[hang,nooneline]{subfigure}

%contols float placements: \FloatBarrier to ensure all floats for a section appear before the next \section 
\usepackage{placeins}

%gestion des éléments graphiques
\usepackage{graphicx}\newcommand{\HRule}{\rule{\linewidth}{0.25pt}}
%-----------------------------------%
%----------listings :---------------%
%------affichage du code-----%
%-----------------------------------%

%importation
\usepackage{listings}

%then use \begin{lstlisting}[style=mypython]
\lstdefinestyle{mypython}{ %
  backgroundcolor=\color{brun},   % couleur de fond
  basicstyle=\footnotesize\ttfamily,,     % taille et police
  belowcaptionskip=3mm,            % modifie la distance entre la légende et le listing
  breakatwhitespace=true,          % coupure uniquement aux espaces
  breaklines=true,                % retour à la ligne automatique
  captionpos=t,                    % position de la légende
  commentstyle=\color{vert},       % couleur des commentaires
  extendedchars=true,              % permets l'affichage de caractères non ASCII
  framerule=0.25pt,				  % épaisseur du cadre
  framesep=1pt,                    % définit la distance entre le cadre et le texte
  frameshape={NNN}{n}{n}{YYY},     % ajoute un cadre autour du code (single, trblTRBL) {<top>}{<left>}{<right>}{<bottom>}, avec Y, N ou R (pour arrondi) pour top et bottom, 3 lettres par étage, lu depuis le centre; y ou n pour left et right, une lettre par étage. Cf listing package doc p. 34.
  framexbottommargin=0pt,          % modifie la marge du bas du contour
  framexleftmargin=0pt,            % modifie la marge de gauche du contour
  framexrightmargin=0pt,           % modifie la marge de droite du contour
  keepspaces=true,                 % garde les espaces dans le code (indentation)
  keywordstyle=\bfseries\color{rouge},       % style des mots clefs scilab
  language=Python,                 % langage
  linewidth=\linewidth,
  numbers=left,                    % emplacement des numéros de ligne
  numbersep=5pt,                   % marge entre les numéros de ligne et le code
  numberstyle=\tt\tiny\color{gris},  % style pour les numéros de ligne
  resetmargins=false,			  % remet ou non les marges à 0 si dans une liste
  rulecolor=\color{bleu},          % couleur du cadre
  rulesep=1pt,                     % définit la distance entre les lignes du cadre
  %rulesepcolor=\color{bleu!50},   % couleur de remplissage entre les doubles lignes du cadre
  showspaces=false,                % ne pas souligner tous les espaces
  showstringspaces=false,          % ne pas souligner les espaces
  showtabs=false,                  % Ne pas indiquer les tabulations dans un string
  stepnumber=1,                    % degré de numérotation des lignes (1 = numérote toutes les lignes)
  stringstyle=\color{orange!50},   % style des éléments 'string'
  tabsize=4,                       % taille des tabulations en espace
  xleftmargin=1pt,                 % modifie la marge de gauche
  xrightmargin=1pt,                % modifie la marge de droite
}
%instantiation des options style général
\lstdefinestyle{Rstyle}{ %
  backgroundcolor=\color{white},   % couleur de fond
  basicstyle=\small\ttfamily,,     % taille et police
  belowcaptionskip=3mm,            % modifie la distance entre la légende et le listing
  breakatwhitespace=true,          % coupure uniquement aux espaces
  breaklines=true,                % retour à la ligne automatique
  captionpos=t,                    % position de la légende
  commentstyle=\color{brun},       % couleur des commentaires
  extendedchars=true,              % permets l'affichage de caractères non ASCII
  framerule=0.25pt,				  % épaisseur du cadre
  framesep=1pt,                    % définit la distance entre le cadre et le texte
  frameshape={NNN}{n}{n}{YYY},     % ajoute un cadre autour du code (single, trblTRBL) {<top>}{<left>}{<right>}{<bottom>}, avec Y, N ou R (pour arrondi) pour top et bottom, 3 lettres par étage, lu depuis le centre; y ou n pour left et right, une lettre par étage. Cf listing package doc p. 34.
  framexbottommargin=0pt,          % modifie la marge du bas du contour
  framexleftmargin=0pt,            % modifie la marge de gauche du contour
  framexrightmargin=0pt,           % modifie la marge de droite du contour
  keepspaces=true,                 % garde les espaces dans le code (indentation)
  keywordstyle=\bfseries\color{bleu},       % style des mots clefs R
  language=R,                 % langage
  linewidth=\linewidth,
  numbers=left,                    % emplacement des numéros de ligne
  numbersep=5pt,                   % marge entre les numéros de ligne et le code
  numberstyle=\tt\tiny\color{darkgrey},  % style pour les numéros de ligne
  resetmargins=false,			  % remet ou non les marges à 0 si dans une liste
  rulecolor=\color{violette},          % couleur du cadre
  rulesep=1pt,                     % définit la distance entre les lignes du cadre
  %rulesepcolor=\color{bleu!50},   % couleur de remplissage entre les doubles lignes du cadre
  showspaces=false,                % ne pas souligner tous les espaces
  showstringspaces=false,          % ne pas souligner les espaces
  showtabs=false,                  % Ne pas indiquer les tabulations dans un string
  stepnumber=1,                    % degré de numérotation des lignes (1 = numérote toutes les lignes)
  stringstyle=\color{orange},   % style des éléments 'string'
  tabsize=4,                       % taille des tabulations en espace
  xleftmargin=1pt,                 % modifie la marge de gauche
  xrightmargin=1pt,                % modifie la marge de droite
}

% le patch suivant corrige un bug concernant la parenthèse fermante et l'option "breaklines"
% source cf. http://tex.stackexchange.com/q/69472   
\usepackage{etoolbox}
\makeatletter
\patchcmd{\lsthk@SelectCharTable}{%
  \lst@ifbreaklines\lst@Def{`)}{\lst@breakProcessOther)}\fi}{}{}{}
\makeatother



%change la numérotation des sous-figures
\renewcommand{\thesubfigure}{\roman{subfigure}.}
\makeatletter
\renewcommand\p@subfigure{\thefigure\,}%ajoute un espace entre le numéro de la figure et celui de la sous-figure
\makeatother
%\captionsetup[subfigure]{labelfont={bf,sf,color=orange},labelformat=simple, labelsep=colon, textfont=normalfont, font=footnotesize}

% change les légendes des listings
\DeclareCaptionFont{white}{\color{white}}
\DeclareCaptionFormat{listing}{{\colorbox{violette!50}{\parbox{\linewidth-6pt}{\hspace{5pt}#1#2#3}}}}
\captionsetup[lstlisting]{format=listing,labelfont=white,textfont=white, singlelinecheck=false, indention=0pt, margin=0pt, font={bf,sf,small}}



% changement du préfixe du float lstlisting (défaut=listing)
\renewcommand{\lstlistingname}{Code} % préfixe dans les légendes de listing
\renewcommand{\lstlistlistingname}{Codes} % préfixe dans les listes de listing

%gestion des commentaires
\usepackage{comment} %use \begin{comment} commentaire \end{comment}
%%gestion de la bibliographie

\usepackage[citestyle=authoryear-comp, backend=biber, bibencoding=inputenc, isbn=false, bibstyle=authoryear, , maxnames=1, language=auto, uniquelist=false, firstinits=true, maxbibnames=99]{biblatex}

%changer l'affichage de la citation
%\renewcommand*{\nameyeardelim}{\addspace}
%\renewcommand*{\postnotedelim}{\addcolon\addspace}
%\renewcommand{\multicitedelim}[0]{\addcomma\addspace}
%\DeclareFieldFormat{postnote}{#1}
%\DeclareFieldFormat{multipostnote}{#1}

%\usepackage{xpatch}
%\xpretobibmacro{date+extrayear}{\addperiod\space}{}{}
%\xapptobibmacro{date+extrayear}{\nopunct}{}{}
%\AtEveryBibitem{%
%\clearname{translator}%
%\clearname{publisher}%
%\clearname{pagetotal}}
%pour citer une page : utiliser \autocite[pagnos]{citekey}
%pour citer plusieur auteurs on utilise \autocites[pagnos]{citekey1}[séparateur][pagnos2]{citekey2}


\renewcommand{\bibnamedash}{\quad\textemdash\quad}%séparateur utilisé pour remplacer le nom de l'auteur
%-------------------------------------
%-------------------------------------


%changer l'affichage de la bibliographie

%\renewcommand{\subtitlepunct}[0]{\addspace\addcolon\addspace}
%the dash to be used as replacement for reccurent author
%\renewcommand{\bibnamedash}{\hspace*{2em}\rule[0.5ex]{2em}{.4pt}\addspace}
%\renewcommand{\labelnamepunct}[0]{\addcolon\addspace}

%-----------remove "in" in journals entries ------%

\renewbibmacro{in:}{%
  \ifentrytype{article}{}{%
  \printtext{\bibstring{in}\intitlepunct}}}
  
%\ExecuteBibliographyOptions{firstinits=true,maxnames=1, maxbibnames=99}

\DeclareNameAlias{sortname}{last-first}

\AtBeginBibliography{%
  \renewcommand*{\multinamedelim}{\addcomma\space}%
}

\DefineBibliographyStrings{english}{%
  andothers = {et\addspace al\adddotspace},%
  andmore = {et\addspace al\adddotspace},%
}
%----------authors in bold-----------%
\usepackage{xpatch}
\xpretobibmacro{author}{\mkbibbold\bgroup}{}{}
\xapptobibmacro{author}{\egroup}{}{}
\xpretobibmacro{bbx:editor}{\mkbibbold\bgroup}{}{}
\xapptobibmacro{bbx:editor}{\egroup}{}{}

%\renewcommand*{\labelnamepunct}{{\addcolon\space}}

%----------remove quotation mark--------%
\DeclareFieldFormat[article]{citetitle}{#1}
\DeclareFieldFormat[article]{title}{#1}
\DeclareFieldFormat[article]{journaltitle}{\textit{#1}}

%----------remove parenthesis and bold for date --------%
\DeclareFieldFormat*{date}{\textmd{#1}}
\DeclareFieldFormat*{parens}{\textmd{#1\addspace\addcolon}}

%-------remove "p." before pages-------------%
\DeclareFieldFormat*{pages}{#1}


%-------using legislation-------------%
\DeclareBibliographyDriver{legislation}{
\ifnameundef{pages}
{\textbf{\printnames{author}} \addspace \printfield{year} \addspace \addcolon \addspace \printfield{title} \addperiod \printfield{journaltitle} \printfield{volume} \addperiod\finentry}
{\textbf{\printnames{author}} \addspace \printfield{year} \addspace \addcolon \addspace \printfield{title} \addperiod \addspace \printfield{journaltitle} \printfield{volume} \addcomma \printfield{pages} \finentry}}


%\bibliography{SBBbib.bib}

\usepackage{rotating,tabularx,booktabs}
\usepackage{tikz}\usepackage{tikz-qtree}\usetikzlibrary{decorations.pathreplacing}
\usepackage{pdflscape, geometry, titling}
%\usepackage{draftwatermark}
\usepackage{translator}
%
\usepackage{siunitx}
\sisetup{
group-separator={\textquoteright}, 
group-minimum-digits = 4,
separate-uncertainty = true, 
list-final-separator = { und }, 
range-phrase = { -- }
%output-decimal-marker={,}%group-digits=true,%group-four-digits=false,% default setting
}
\DeclareSIUnit[number-unit-product = \;]\CHF{ CHF }
\DeclareSIPrefix\Mio{Mio }{6}
\IfFileExists{upquote.sty}{\usepackage{upquote}}{}

 \pgfplotsset{compat=1.12}