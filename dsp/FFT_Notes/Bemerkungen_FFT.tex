\documentclass[11pt,a4paper,DIV12,BCOR1.5mm]{scrartcl}
\usepackage[utf8]{inputenc}
\usepackage[T1]{fontenc}
%\usepackage[ngerman]{babel}
\usepackage{amsmath}
\usepackage{amsfonts}
\usepackage{amssymb}

%%%%%%%%%%%%%%%%%%%%%%%%%%%%%%%%%
\author{Enzo Scossa-Romano}
\date{24 Juli 2014}
\title{Signal Processing}

\begin{document}
%%%%%%%%%%%%%%%%%%%%%%
%Titolo%%%%%%%%%%%%%%%
%%%%%%%%%%%%%%%%%%%%%%
\maketitle
\tableofcontents
%%%%%%%%%%%%%%%%%%%%%
\section{DTFT}
la discret time fourier trasform é l'equivalente della DFT per sequenze infinite. Solitamente viene definita prima della DFT ed é data da 
\begin{equation}
X(\omega) = \sum_{n=-\infty}^{\infty} x(n)\cdot \exp^{-i\,n\,\omega},\qquad \omega \in -\pi,\pi
\end{equation} 
Le frequenze sono continue. La relazione tra DFT e DTFT la si può osservare facendo la DTFT di un segnale moltiplicato con una finestra rettangolare. 
\begin{itemize}
	\item eigenvectors of linear convolution
	\item $2\pi$ periodic
\end{itemize}

\section{DFT/FFT: dalla matematica alla pratica}
Sequenza di $N$ elementi $x= {x(n)}_{n=0,...,N-1}$. La DFT é:
\begin{equation}
	X(k) = \sum_{n=0}^{N-1} x(n)\cdot \exp^{-i\,n\,k \frac{2\pi}{N}},\qquad k = 0..N-1
\end{equation}
FFT é il nome dell'algoritmo numerico per calcolare la DFT.\\

\begin{itemize}
\item 
	Sequenza $X$ ha nuovamente $N$ Elementi complessi. $X(k) = |X(k)| \cdot \exp{i\phi}$ dove $\phi$ rappresenta la fase. La fase é determinata fino a modulo $2\pi$. 
\item 
	Con il prodotto scalare $ <x,y> =  \sum_{n=0}^{N-1} x(n)\cdot y(n)^*$  e la Base ortogonale $y_k(n)= e^{i\,n\,k \frac{2\pi}{N}}$ come un Cambiamento di base.
\item 
	per la base $y_k$ vale: $y_k = y_{k+lN}$  e $ y_{N-k} = y_{-k} = y_{k}^* $. Il modulo é $|y_k|=N$.\\
	Una base equivalente é quindi rappresentata da $\{(y_k, y_{k}^*)\}$ per $ k = 0..(N)/2$  
\item 
	$X(k)$ é la proiezione di $x$ su $y_k$.
\item 
	se $x \in \mathcal{R}$  Vale $X(k)= X(N-k)^*$ e $X(N-k) = |X(k)|\cdot \exp{i(2\pi-\phi)} $  
\item 
	Nella nuova base il prodotto scalare diventa $<X,Y> = \frac{1}{N}\sum_{n=0}^{N-1} Y(k)\cdot Y(k)^*$. Ne segue Parseval.
\item 
	Ogni base $y_k$ é caratterizzata dalla Frequenza 
	\begin{equation}
	f=|k|/N \in\{0,1/N,2/N,...,(N-1)/N\} 
	\end{equation}
	Dalle proprietà sopra elencate si ottengono le seguenti interpretazioni: le frequenze rilevanti sono solamente $N/2$ ($(N-1)/2$ se $N$ dispari) con frequenza massima uguale a $\frac{1}{2}$. Le frequenze $k/N$ e $(k+lN)/N$ sono le stesse. Le basi $y_k$, $y_{N-k}$ o $y_{-k}$ hanno la stessa frequenza.
\item 
	La DFT per una sequenza finita ha frequenze discrete e finite. Le frequenze sono determinate dalla discretizazione del tempo (limite superiore) e date dalla lunghezza del segnale(risoluzione).
	
\item la DFT di un vettore a N dimensioni può essere calcolata per qualsiasi frequenza $\omega_k$. Questo procedimento corrisponde a 0-padding
\end{itemize}

\subsection{Relazione tra DTFT e DFT}
nella DFT $x(n)$ ha un numero di elementi finiti. Per paragonare le due sequenze il segnale deve essere esteso ad un numero infinito di elementi. Questo può essere fatto in due modi:
\begin{enumerate}
	\item 0-Padding denominata con $\hat x$
	\item circular extension. $\tilde x (n) = x( n \, mod_N)$. Il segnale esteso all infinito é rappresebtabile da $\frac{1}{N}\sum_k ^{N-1} X(k) \exp^{i\omega_k\,n}$.
	Ogni segnale (infinito) periodico (periodo finito di lunghezza N) é rappresentabile in questo modo. 
\end{enumerate}

The DFT computes the DTFT of a finite segment of an infinite-length signal at a finite number of frequencies; it is thus the operational tool for computing the DTFT, which cannot be computed in full.

\textbf{caso 1, DFT approssima la DTFT di $\tilde{x}$}
\begin{itemize}
\item 
	la DFT equivale alla DTFT dello spezzone zero padded  campionata alle frequenze $\omega_k$
\item 
	se nella DFT si utilizza uno 0-padding si aumentala risoluzione della campionatura.
	Se il segnale infinito é realmente 0 al di furi di $x(n)$ tramite zero padding si aumentano le informazioni ricavate con la DFT.
\item 
	Aumentando la lunghezza $T$ del segnale si aumenta la risoluzione della frequenza nella sua DTFT. Analogamente aumentando l'intervallo delle frequenze conosciute si aumenta la risoluzione temporale del segnale. 
\item 
	la ricostruzione del segnale con la DFT sarà periodico(se esteso all infinito)

\end{itemize}

\textbf{caso 2: DFT equivalente alla DTFT di $\tilde{x}$}
\begin{itemize}
\item 
	In questo caso la DTFT é una sommatoria di funzioni delta. Al massimo in $N$ punti é diversa da zero. in quei punti diverge.
\item la relazione tra DFT e DTFT é
 	\begin{equation}
 	 \tilde X(\omega) = \sum_k^{N-1} \frac{2\pi}{N} X(k) \cdot \delta(\omega -k\cdot \frac{2\pi}{N})
 	\end{equation}

\end{itemize}

\textbf{caso3: relazione tra DTFT di $\tilde x $ e $x$ 0-padded}

\begin{align}
\tilde{X}(\omega) =& \sum_{n=-\infty}^\infty \tilde{x}(n) \exp^{-i\omega n} \\
=&  \sum_{l=-\infty}^\infty \sum_{n=0}^{N-1} \tilde{x}(lN+n) \exp^{-i\omega (lN+n)} \\
=&  \sum_{l=-\infty}^\infty \sum_{n=0}^{N-1} x(n) \exp^{-i\omega n} \exp^{-i\omega Nl} \\
=&  \hat X(\omega) \cdot \sum_{l=-\infty}^\infty    \exp^{-i\omega Nl}
\end{align}

$ \sum_{l=-\infty}^\infty    \exp^{-i\omega N\, l} $ é uguale a 0 se $\omega N = 2\pi k$ altrimenti a $ $

Spesso il segnale reale $x(n)$ é infinito e non periodico. \textbf{Si é interessati alla DTFT del segnale ma questo non si può calcolare}. Quello che viene fatto é prendere uno spezzone del segnale infinito di lunghezza N $x_N(n)$. Perndere lo spezzone equivale a moltiplicare il segnale con una finestra; il segnale viene trasformato in un segnale infinito $\tilde x_N(n)$ 0-padded. Con la DFT si può ottenere un'approssimazione della DTFT del segnale $\tilde x_N(n)$ in n punti.  Per ricollegarsi alla DTFT del segnale originale é necessario capire la relazione tra le DTFT dei due segnali infiniti $x(n)$ e $\tilde x_N(n)$ .\\



\subsection{Dalla sequenza matematica al segnale reale}

Fino ad adesso tutte le discussioni su FFT sono matematiche. Per passare ad un segnale fisico bisogna considerare la durata totale $T$ in secondi della sequenza $x$ o in modo equvalente la sampling rate $Fs$ oppure ancora il lasso di tempo $\Delta t$ tra due elementi della sequenza. La relazione tra queste grandezze é:
\begin{equation}
	Fs =  \frac{N}{T} = \frac{1}{ \Delta t }\,.
\end{equation}
Le considerazioni Matematiche sono come se la durata fosse $T=N$ , $Fs = 1$ e $\Delta t = 1$\\
La frequenza corrispondente a $k$ si ottiene moltiplicando per $Fs$ 
\begin{equation}
	f_k = \frac{|k|\cdot Fs}{N} = \frac{|k|}{N \cdot \Delta t}= \frac{|k|}{ T }
\end{equation} 
le frequenze sono spaziate equidistantemente con distanza $\Delta f=\frac{1}{N \Delta t} =\frac{1}{T} $ 
la frequenza max $f_{k=N/2} = \frac{Fs}{2} $ ha periodo  $\frac{2T}{N}=\frac{2}{Fs} $, la minima con $f_1=\Delta f=\frac{1}{T}$ ha periodo $T$. Da notare che $\Delta f$ dipende solo da $T$.

\section{Convoluzione}
la convoluzione é un operazione $(C^n\times C^n \mapsto C^n)$ tra due sequenze.
Differenziamo due tipi di convoluzioni
\subsection{convoluzione lineare per sequenze infinite}
definita nel seguente modo
\begin{equation}
(h\ast x)(n) = \sum_s h(s)\cdot x(n-s)
\end{equation}
la somma é infinita da $-\inf$ fino a $+\infty$.\\
La convoluzione é bilineare e simmetrica (rappresentabile con un tensore $T\in \varLambda^1_2$.
\begin{itemize}
	\item se le sequenze sono finite possono essere 0-padded
	\item il risultato é sempre definito per sequenze infinite
	\item se una delle due sequenze é periodica il risultato sarà periodico
	\item il tensore $T$ nella base $\delta_i$ é rappresentato da $T_{i,j}^n = \delta_{n-(i+j)}$
	\item  La si può comporre con una inversione del tempo R e un shift di n  elementi $S(n)$.\\
	Inversione del tempo $y = R x$ con $y(n) = x(-n) = R^n_j x(j) =  \delta ^n_{j+n} x(j)$. segue che R nelle base standard é rappresentato da $\delta ^n_{j+n}$\\
	shift del tempo $y = S_l x$ con $y(n) = x(n-l) = (S_l)^n_j x(j) =  \delta ^n_{j-(n-l)} x(j)$. Segue che $S_l$ nelle base standard é rappresentato da $\delta^n_{j-(n-l)}$
	
	\item nella base $v_\omega (n) = \exp^{-i\omega\cdot n}$ otteniamo
	\begin{equation}
	T(v_\omega,v_{\omega'}) = \sum_k \exp^{-i\omega\cdot k} \exp^{-i\omega'\cdot (n-k)} = \sum_k \exp^{-i(\omega-\omega')\cdot k} \exp^{-i\omega'\cdot n} = v_\omega 
	\end{equation} 
	se $\omega = \omega'$ altrimenti 0.
	
	\item  nella base di furrier DTFT (infinita) questa operazione é diagonalizzata, vale quindi:
	\begin{equation}
	DTF[(h\ast x)(n)](k) = H(k) \cdot X(k)
	\end{equation}
	Da notare che se si definisce operazionalmente la stessa operazione nella base di fourier si ottiene la stessa proprietà nella base 'euclidea'.
\end{itemize}


\subsection{convoluzione ciclica per sequenze finite e infinite periodiche}
definita nel seguente modo per sequenze finite di lunghezza N
\begin{equation}
(h \circledast_N x)(n) = \sum_{s=0}^{N-1} h(s)\cdot x(n-s)_{mod_N}
\end{equation}
La sommatoria in questo caso é finita. Si possono calcolare infiniti elementi $n$, ma in ogni caso il risultato ha periodo N.\\

La domanda fondamentale consiste nel determinare quando le due definizioni sono equivalenti.

\begin{itemize}
	\item nel caso $x(n)$ sia N-periodico sostituendo la sequenza $h$ (infinita) con $h_N$ (periodica)
	\begin{equation}
	h_N(n) = \sum_{s=-\infty}^{\infty} h(n+sN)
	\end{equation}
	si ottiene $(h_N \circledast_N x)(n) = (h\ast x)(n) $. Ricordare che il risultato é periodico in entrambi i casi.
	\item se $h$ é diverso da zero per soli finiti elementi, schegliendo N sufficientemente grande si ottiene $h_N = h$.
	\item nel caso che $h$ e $x$ siano finiti di lunghezza $M$ e $L$ abbiamo che la convoluzione lineare delle sequenze 0-padded é uguale alla convoluzione ciclica nel caso in cui $N = M+L-1$
	
	\item  nella base di furrier DFT/FFT(infinita) questa operazione é diagonalizzata

	\item si sceglie la sequenza più lunga (M). si pone $N=2M$. si 0-pad entrabi le sequenze alla lunghezza $N$. la convoluzione 0 padded é ugusale alla convoluzione lineare(con sequenze 0-padded) per gli indici $n \in 0..N$. lineare é altrimenti 0, la circolare periodica.
\end{itemize}


\section{Windowing}
 Windowing é un procedimento per ottenere da sequenze infinite generiche sequenze Infinite con solo $N$ elementi dieversi da 0.  La sequenza si può quindi trattare analogamente come sequenza infinita zero padded che a sua volta viene poi trattata con FFT che calcola i valori equivalenti ad una sequenza finita periodicizzata.\\
 il procedimento  é il seguente:
 \begin{equation}
 x_w(n) = w(n)\cdot x(n)
 \end{equation}
 
 
 vale 
  \begin{equation}
  X_w (\omega) = (W \ast X) (\omega)
  \end{equation}
  
  \subsection{Energia}
  
  
 
 
 
 
\section{Zero padding}
Zero padding una sequenza significa aggiungere 0 alla fine di questultima. calcolando la DTDT di una sequenza zero padded si ottengono le seguenti proprietà

\begin{itemize}
\item aggiungere zeri  all'inizio piuttosto che alla fine ha come unico effetto nella DTFT del segnale uno shift nella fase
\item ...
\end{itemize}

\section{Energia e Spettri}
Per gli spettri ci sono molte definizioni e alcune differenze. Bisogna differenziare tra sequenze  discrete finite, discrete infinite, continue finite e continue infinite
\begin{itemize}
	\item  nel caso di sequenze discrete infinite (DTFT) e continue (FT) si parla di densità per lo spettro 
	\item   nel caso di sequenze discrete finite (DFT) e continue  finite/ periodiche(FT) lo spettro é sotto forma di serie
	\item casi con spettri discreti possono essere trasformati in casi continui con appropriate estensioni.
	\item  quando si parla di autospectrum si intende la DFT FT o DTFT della correlazione
	
	\item  quando si parla di powerspectrum si l'onesided autospectrum
	
	\item  quando si parla di (amplitude) spectrum si intende la DFT FT o DTFT della sequenza
	\item  quando si parla di Energispectrum si intende la DFT FT o DTFT della correlazione
	
\end{itemize} 
\subsection{Energia matematica}
L'energia di una sequenza lunga $N$ punti o infinita é definita come:
\begin{equation}
E_m[x] = \sum_{n=0}^{N-1} x(n)\cdot x(n)^* = |x|^2
\end{equation}
Con parseval vale
\begin{itemize}
	\item \textit{sequenza infinita; DTFT}
	\begin{equation}
	E[x]_m =  <x,x> = <X,X> =   \frac{1}{2\pi} \int_{-\pi}^{\pi} |X(\omega)|^2d\,\omega
	\end{equation}
	\item \textit{sequenzafinita;DFT}
	\begin{equation}
	E[x]_m =  <x,x> = \frac{1}{N} <X,X> =   \frac{1}{N} \sum_k |X(k)|^2
	\end{equation}
\end{itemize}

\subsection{Deterministic correlation and autocorrelation per sequenze infinite}
la correlazione deterministica tra due sequenze(infinite) é definita come
\begin{equation}
 (x\star y)(n)=  \sum_k x(k) y^*(k-n) \qquad \textrm{shift e prodotto scalare}
\end{equation}
La autocorrelazione di una sequenza $x$
\begin{equation}
a_x(n)=  \sum_k x(k) x^*(k-n) 
\end{equation}
che si può scrivere come convoluzione tra $x(n)$ e $x^*(-n)$. da notare che vale $a_x(0) = E[x]_m $. \\
La DTFT della autocorrelazione é reale e simmetrica (anche per sequenze complesse) e definisce in modo generale il concetto di \textbf{Powerspectrumdensity}(one sided).
\begin{equation}
A_x(\omega) = |X(\omega)|^2\ =\frac{1}{2} S(\omega),.
\end{equation}
la inversa DTFT per tempo $t=0$  da come risultato, equivalentemente a parseval
\begin{equation}
E[x]_m =  \frac{1}{2\pi}\int_0^\pi\,S(\omega)\,d\omega,.
\end{equation}
\subsection{Autospectrum per sequenze finite o infinite (con N elementi non-0)}
Analogamente a quanto fatto per le sequenze infinite , utilizzando le definizioni circolari si  può definire energia \textbf{powerspectrum} per sequenze finite $x(n)$. Denotiamo $\hat{x}$  una qualunque estensione  di $x$ finita (0-padded) di lunghezza $Nz$.\\
\begin{equation}
E[\hat{x}]_m = a_{\hat{x}}(0) \qquad \textrm{convoluzione circolare}
\end{equation}
e
\begin{equation}
\frac{1}{2}S_{\tilde{x}}(\omega_k) = |X(k)|^2 =  A_{\hat{x}}(k)
\end{equation}

Definiamo $\tilde x(n) $ la sequenza infinita 0-padded  di $x$.  Valgono le seguenti  proprietà:\\
 
L'uguaglianza 
\begin{equation}
a_{\tilde{x}}(n) =  a_{\hat{x}}(n) \textrm{lineare, circolare}
\end{equation}
 é valida se $Nz>=2N-1$. Per l'elemento $a(n=0)$ l'uguaglianza é sempre valida (vedi anche prodotto scalare)
\begin{equation}
E[\tilde{x}]_m = a_{\tilde{x}}(0) =  a_{\hat{x}}(0) = E[\hat{x}]_m 
\end{equation}
quindi anche (parseval o inverse DTFT)
\begin{equation}
E[\tilde{x}]_m =  \frac{1}{2\pi}\int_0^\pi\,S_{\tilde{x}}(\omega)\,d\omega = E[\hat{x}]_m =  \frac{1}{Nz}\sum_k^{Nz/2}\,S_{\hat{x}}(\omega_k)\,.
\end{equation}

\begin{equation}
E[\tilde{x}]_m \overset{f = \frac{n_s \omega}{2\pi}}{=} \frac{1}{n_s}\int_0^\frac{n_s}{2}\,S_{\tilde{x}}(\omega_f)\,df = E[\hat{x}]_m =  \frac{1}{n_s}\sum_k^{Nz/2}\,S_{\hat{x}}(\omega_k)\Delta f\,.
\end{equation}
Per lo spettro vale che il caso finito é una campionatura del caso infinito se $Nz>=2N-1$.(da dft e dtft definizione)
\begin{equation}
S_{\tilde{x}}(\omega_k) = S_{\hat{x}}(\omega_k) \qquad 
\end{equation}


\subsection{Calcolo di $p_{rms}^2$, power spectral density (PSD), Energy spectral density (ESD) }
partiamo da un segnale continuo $p(t)$ diverso da 0 per un intervallo finito $T$ (esempio un rumore di n secondi). $p_{rms}^2$ é definito  nel seguente modo:
\begin{equation}
	p_{rms}^2 = \frac{1}{T} \int_{-\frac{T}{2}}^{\frac{T}{2}} p^2(t) dt 
\end{equation}
l'integrale si può estendere a infinito nel caso di una sequenza infinita. La discretizzazione del segnale porta al segnale finito $p(n)$ (senza windowing). L'integrale lo si può approssimare con 
\begin{equation}
p_{rms}^2 \approx \frac{1}{T} \sum_{n}^{Nz} \hat{p}^2(n) \Delta t = \frac{1}{T} E[\hat p]_m \Delta t = \frac{1}{T} a_{\hat{p}}(0) \Delta t = \frac{E[\hat p]_m}{N}= \frac{a_{\hat{p}}(0)}{N} 
\end{equation}
per qualsiasi  estensione (finita/infinita) $\hat p(n)$. Attenzione $T$ rimane lo stesso per qualunque estensione $\hat p$. $prms$ é indipendente dallo time scaling (frequenza).
\begin{equation}
p_{rms}^2  = \frac{\Delta t^2}{T}\int_0^\frac{n_s}{2}\,S_{\tilde{x}}(\omega_f)\,df = \frac{\Delta t^2}{T}\sum_k^{Nz/2}\,S_{\hat{x}}(\omega_k)\Delta f\,.
\end{equation}
con $\frac{\Delta t^2}{T} = \frac{\Delta t}{N}$.
relazione tra prms e autocorrelation / energia
autocorrelation
autopowerspectrum

caso segnale con finiti elementi diversi da 0 tramite windowing($p_{rms}^2(t)$)



\end{document}
